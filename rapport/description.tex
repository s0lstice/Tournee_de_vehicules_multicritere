\chapter{Introduction}

Pour ce projet nous intéressons à réaliser un programme qui permet en fonction d'une liste de lieux de ressortir la liste des différents parcours possibles pour réaliser cette tournée entre les différents lieux. Ce problème est un peu similaire à celui du voyageur de commerce. 
Le principe de l'application qu'on essaye de développer est de permettre à partir d'une Base de données représentant la configuration d'une ville.
C'est-à-dire un ensemble de lieux relié entre par des arcs, de calculer un parcours pour passer par tous les lieux en fonction de leur intérêt pour l'utilisateur.
La difficulté majeure est de retourner des solutions dans une  période très courte. 
\begin{figure}[h]
\begin{center}
%LaTeX with PSTricks extensions
%%Creator: inkscape 0.48.0
%%Please note this file requires PSTricks extensions
\psset{xunit=.5pt,yunit=.5pt,runit=.5pt}
\begin{pspicture}(430.77990723,228.01989746)
{
\newrgbcolor{curcolor}{1 1 1}
\pscustom[linestyle=none,fillstyle=solid,fillcolor=curcolor]
{
\newpath
\moveto(229.80462576,184.35594205)
\curveto(229.80462576,170.40864637)(218.49810865,159.10212926)(204.55081297,159.10212926)
\curveto(190.60351729,159.10212926)(179.29700018,170.40864637)(179.29700018,184.35594205)
\curveto(179.29700018,198.30323773)(190.60351729,209.60975484)(204.55081297,209.60975484)
\curveto(218.49810865,209.60975484)(229.80462576,198.30323773)(229.80462576,184.35594205)
\closepath
}
}
{
\newrgbcolor{curcolor}{0 0 0}
\pscustom[linewidth=1,linecolor=curcolor]
{
\newpath
\moveto(229.80462576,184.35594205)
\curveto(229.80462576,170.40864637)(218.49810865,159.10212926)(204.55081297,159.10212926)
\curveto(190.60351729,159.10212926)(179.29700018,170.40864637)(179.29700018,184.35594205)
\curveto(179.29700018,198.30323773)(190.60351729,209.60975484)(204.55081297,209.60975484)
\curveto(218.49810865,209.60975484)(229.80462576,198.30323773)(229.80462576,184.35594205)
\closepath
}
}
{
\newrgbcolor{curcolor}{1 1 1}
\pscustom[linestyle=none,fillstyle=solid,fillcolor=curcolor]
{
\newpath
\moveto(426.78436516,184.35594435)
\curveto(426.78436516,170.40864867)(415.47784805,159.10213156)(401.53055237,159.10213156)
\curveto(387.58325669,159.10213156)(376.27673958,170.40864867)(376.27673958,184.35594435)
\curveto(376.27673958,198.30324003)(387.58325669,209.60975714)(401.53055237,209.60975714)
\curveto(415.47784805,209.60975714)(426.78436516,198.30324003)(426.78436516,184.35594435)
\closepath
}
}
{
\newrgbcolor{curcolor}{0 0 0}
\pscustom[linewidth=1,linecolor=curcolor]
{
\newpath
\moveto(426.78436516,184.35594435)
\curveto(426.78436516,170.40864867)(415.47784805,159.10213156)(401.53055237,159.10213156)
\curveto(387.58325669,159.10213156)(376.27673958,170.40864867)(376.27673958,184.35594435)
\curveto(376.27673958,198.30324003)(387.58325669,209.60975714)(401.53055237,209.60975714)
\curveto(415.47784805,209.60975714)(426.78436516,198.30324003)(426.78436516,184.35594435)
\closepath
}
}
{
\newrgbcolor{curcolor}{1 1 1}
\pscustom[linestyle=none,fillstyle=solid,fillcolor=curcolor]
{
\newpath
\moveto(319.7081966,178.80009773)
\curveto(319.7081966,177.96325994)(304.10520297,177.28486887)(284.85793491,177.28486887)
\curveto(265.61066686,177.28486887)(250.00767323,177.96325994)(250.00767323,178.80009773)
\curveto(250.00767323,179.63693553)(265.61066686,180.3153266)(284.85793491,180.3153266)
\curveto(304.10520297,180.3153266)(319.7081966,179.63693553)(319.7081966,178.80009773)
\closepath
}
}
{
\newrgbcolor{curcolor}{1 1 1}
\pscustom[linestyle=none,fillstyle=solid,fillcolor=curcolor]
{
\newpath
\moveto(228.79446856,25.76199255)
\curveto(228.79446856,11.81469687)(217.48795145,0.50817976)(203.54065577,0.50817976)
\curveto(189.59336009,0.50817976)(178.28684298,11.81469687)(178.28684298,25.76199255)
\curveto(178.28684298,39.70928823)(189.59336009,51.01580534)(203.54065577,51.01580534)
\curveto(217.48795145,51.01580534)(228.79446856,39.70928823)(228.79446856,25.76199255)
\closepath
}
}
{
\newrgbcolor{curcolor}{0 0 0}
\pscustom[linewidth=1,linecolor=curcolor]
{
\newpath
\moveto(228.79446856,25.76199255)
\curveto(228.79446856,11.81469687)(217.48795145,0.50817976)(203.54065577,0.50817976)
\curveto(189.59336009,0.50817976)(178.28684298,11.81469687)(178.28684298,25.76199255)
\curveto(178.28684298,39.70928823)(189.59336009,51.01580534)(203.54065577,51.01580534)
\curveto(217.48795145,51.01580534)(228.79446856,39.70928823)(228.79446856,25.76199255)
\closepath
}
}
{
\newrgbcolor{curcolor}{1 1 1}
\pscustom[linestyle=none,fillstyle=solid,fillcolor=curcolor]
{
\newpath
\moveto(426.78436516,25.76199255)
\curveto(426.78436516,11.81469687)(415.47784805,0.50817976)(401.53055237,0.50817976)
\curveto(387.58325669,0.50817976)(376.27673958,11.81469687)(376.27673958,25.76199255)
\curveto(376.27673958,39.70928823)(387.58325669,51.01580534)(401.53055237,51.01580534)
\curveto(415.47784805,51.01580534)(426.78436516,39.70928823)(426.78436516,25.76199255)
\closepath
}
}
{
\newrgbcolor{curcolor}{0 0 0}
\pscustom[linewidth=1,linecolor=curcolor]
{
\newpath
\moveto(426.78436516,25.76199255)
\curveto(426.78436516,11.81469687)(415.47784805,0.50817976)(401.53055237,0.50817976)
\curveto(387.58325669,0.50817976)(376.27673958,11.81469687)(376.27673958,25.76199255)
\curveto(376.27673958,39.70928823)(387.58325669,51.01580534)(401.53055237,51.01580534)
\curveto(415.47784805,51.01580534)(426.78436516,39.70928823)(426.78436516,25.76199255)
\closepath
}
}
{
\newrgbcolor{curcolor}{1 1 1}
\pscustom[linestyle=none,fillstyle=solid,fillcolor=curcolor]
{
\newpath
\moveto(51.00762516,184.35594205)
\curveto(51.00762516,170.40864637)(39.70110805,159.10212926)(25.75381237,159.10212926)
\curveto(11.80651669,159.10212926)(0.49999958,170.40864637)(0.49999958,184.35594205)
\curveto(0.49999958,198.30323773)(11.80651669,209.60975484)(25.75381237,209.60975484)
\curveto(39.70110805,209.60975484)(51.00762516,198.30323773)(51.00762516,184.35594205)
\closepath
}
}
{
\newrgbcolor{curcolor}{0 0 0}
\pscustom[linewidth=1,linecolor=curcolor]
{
\newpath
\moveto(51.00762516,184.35594205)
\curveto(51.00762516,170.40864637)(39.70110805,159.10212926)(25.75381237,159.10212926)
\curveto(11.80651669,159.10212926)(0.49999958,170.40864637)(0.49999958,184.35594205)
\curveto(0.49999958,198.30323773)(11.80651669,209.60975484)(25.75381237,209.60975484)
\curveto(39.70110805,209.60975484)(51.00762516,198.30323773)(51.00762516,184.35594205)
\closepath
}
}
{
\newrgbcolor{curcolor}{0 0 0}
\pscustom[linestyle=none,fillstyle=solid,fillcolor=curcolor]
{
\newpath
\moveto(199.28991123,185.14973763)
\lineto(199.28991123,174.46614388)
\lineto(205.61803623,174.46614388)
\curveto(207.74041574,174.46614063)(209.30942459,174.90233812)(210.32506748,175.77473763)
\curveto(211.35369338,176.66014886)(211.86801578,178.00780376)(211.86803623,179.81770638)
\curveto(211.86801578,181.64061263)(211.35369338,182.98175712)(210.32506748,183.84114388)
\curveto(209.30942459,184.71352622)(207.74041574,185.1497237)(205.61803623,185.14973763)
\lineto(199.28991123,185.14973763)
\moveto(199.28991123,197.14192513)
\lineto(199.28991123,188.35286263)
\lineto(205.12975498,188.35286263)
\curveto(207.05682268,188.3528455)(208.48911291,188.71091806)(209.42662998,189.42708138)
\curveto(210.37713186,190.15622911)(210.8523918,191.26299884)(210.85241123,192.74739388)
\curveto(210.8523918,194.21872505)(210.37713186,195.31898437)(209.42662998,196.04817513)
\curveto(208.48911291,196.77731624)(207.05682268,197.14189921)(205.12975498,197.14192513)
\lineto(199.28991123,197.14192513)
\moveto(195.34459873,200.38411263)
\lineto(205.42272373,200.38411263)
\curveto(208.43051922,200.38408347)(210.74822524,199.75908409)(212.37584873,198.50911263)
\curveto(214.00343031,197.25908659)(214.81723158,195.48174462)(214.81725498,193.17708138)
\curveto(214.81723158,191.39320704)(214.40056533,189.97393763)(213.56725498,188.91926888)
\curveto(212.73390033,187.86456474)(211.50994322,187.20701331)(209.89537998,186.94661263)
\curveto(211.83546373,186.52993065)(213.33936848,185.65753569)(214.40709873,184.32942513)
\curveto(215.48780383,183.01430917)(216.02816787,181.3671754)(216.02819248,179.38801888)
\curveto(216.02816787,176.78384665)(215.14275209,174.77212991)(213.37194248,173.35286263)
\curveto(211.60108897,171.93359108)(209.08156024,171.22395638)(205.81334873,171.22395638)
\lineto(195.34459873,171.22395638)
\lineto(195.34459873,200.38411263)
}
}
{
\newrgbcolor{curcolor}{0 0 0}
\pscustom[linestyle=none,fillstyle=solid,fillcolor=curcolor]
{
\newpath
\moveto(401.11339756,197.78645638)
\curveto(399.08213688,197.78642981)(397.55219049,196.78382665)(396.52355381,194.77864388)
\curveto(395.5079217,192.78643481)(395.00010971,189.78513573)(395.00011631,185.77473763)
\curveto(395.00010971,181.77733124)(395.5079217,178.77603216)(396.52355381,176.77083138)
\curveto(397.55219049,174.77864032)(399.08213688,173.78254757)(401.11339756,173.78255013)
\curveto(403.15765363,173.78254757)(404.68760002,174.77864032)(405.70324131,176.77083138)
\curveto(406.73186881,178.77603216)(407.24619121,181.77733124)(407.24621006,185.77473763)
\curveto(407.24619121,189.78513573)(406.73186881,192.78643481)(405.70324131,194.77864388)
\curveto(404.68760002,196.78382665)(403.15765363,197.78642981)(401.11339756,197.78645638)
\moveto(401.11339756,200.91145638)
\curveto(404.38161074,200.91142669)(406.87509783,199.61585507)(408.59386631,197.02473763)
\curveto(410.32561522,194.4465894)(411.19149977,190.69659315)(411.19152256,185.77473763)
\curveto(411.19149977,180.86587382)(410.32561522,177.11587757)(408.59386631,174.52473763)
\curveto(406.87509783,171.9466119)(404.38161074,170.65755069)(401.11339756,170.65755013)
\curveto(397.84515895,170.65755069)(395.34516145,171.9466119)(393.61339756,174.52473763)
\curveto(391.89464406,177.11587757)(391.03526992,180.86587382)(391.03527256,185.77473763)
\curveto(391.03526992,190.69659315)(391.89464406,194.4465894)(393.61339756,197.02473763)
\curveto(395.34516145,199.61585507)(397.84515895,200.91142669)(401.11339756,200.91145638)
}
}
{
\newrgbcolor{curcolor}{0 0 0}
\pscustom[linestyle=none,fillstyle=solid,fillcolor=curcolor]
{
\newpath
\moveto(26.29369923,199.52786568)
\lineto(20.94213673,185.01614693)
\lineto(31.66479298,185.01614693)
\lineto(26.29369923,199.52786568)
\moveto(24.06713673,203.41458443)
\lineto(28.53979298,203.41458443)
\lineto(39.65307423,174.25442818)
\lineto(35.55151173,174.25442818)
\lineto(32.89526173,181.73489693)
\lineto(19.75073048,181.73489693)
\lineto(17.09448048,174.25442818)
\lineto(12.93432423,174.25442818)
\lineto(24.06713673,203.41458443)
}
}
{
\newrgbcolor{curcolor}{0 0 0}
\pscustom[linestyle=none,fillstyle=solid,fillcolor=curcolor]
{
\newpath
\moveto(194.3344669,39.76988533)
\lineto(212.7719669,39.76988533)
\lineto(212.7719669,36.44957283)
\lineto(198.2797794,36.44957283)
\lineto(198.2797794,27.81676033)
\lineto(212.16649815,27.81676033)
\lineto(212.16649815,24.49644783)
\lineto(198.2797794,24.49644783)
\lineto(198.2797794,13.93004158)
\lineto(213.1235294,13.93004158)
\lineto(213.1235294,10.60972908)
\lineto(194.3344669,10.60972908)
\lineto(194.3344669,39.76988533)
}
}
{
\newrgbcolor{curcolor}{0 0 0}
\pscustom[linestyle=none,fillstyle=solid,fillcolor=curcolor]
{
\newpath
\moveto(410.1197147,37.52376106)
\lineto(410.1197147,33.36360481)
\curveto(408.79156527,34.60055999)(407.37229585,35.52503823)(405.8619022,36.13704231)
\curveto(404.36448636,36.74899534)(402.76943587,37.05498462)(401.07674595,37.05501106)
\curveto(397.74339923,37.05498462)(395.19131845,36.03285022)(393.42049595,33.98860481)
\curveto(391.64965532,31.95733347)(390.76423954,29.01462808)(390.76424595,25.16047981)
\curveto(390.76423954,21.31932327)(391.64965532,18.37661788)(393.42049595,16.33235481)
\curveto(395.19131845,14.30110112)(397.74339923,13.28547714)(401.07674595,13.28547981)
\curveto(402.76943587,13.28547714)(404.36448636,13.59146642)(405.8619022,14.20344856)
\curveto(407.37229585,14.81542353)(408.79156527,15.73990177)(410.1197147,16.97688606)
\lineto(410.1197147,12.85579231)
\curveto(408.73948198,11.91829101)(407.2746397,11.21516671)(405.72518345,10.74641731)
\curveto(404.18870529,10.27766765)(402.56110275,10.04329288)(400.84237095,10.04329231)
\curveto(396.42829638,10.04329288)(392.95173736,11.39094778)(390.41268345,14.08626106)
\curveto(387.87361743,16.79458821)(386.60408745,20.48599077)(386.6040897,25.16047981)
\curveto(386.60408745,29.84796058)(387.87361743,33.53936313)(390.41268345,36.23469856)
\curveto(392.95173736,38.94300356)(396.42829638,40.29716888)(400.84237095,40.29719856)
\curveto(402.58714439,40.29716888)(404.22776775,40.06279411)(405.76424595,39.59407356)
\curveto(407.31370216,39.13831587)(408.76552363,38.44821239)(410.1197147,37.52376106)
}
}
{
\newrgbcolor{curcolor}{0 0 0}
\pscustom[linewidth=1,linecolor=curcolor]
{
\newpath
\moveto(203.540653,159.10212746)
\curveto(235.176043,122.53003746)(241.810173,86.91954746)(203.540653,53.03610746)
\curveto(173.122023,81.09097746)(166.743023,114.91535746)(203.540653,159.10212746)
\closepath
}
}
{
\newrgbcolor{curcolor}{0 0 0}
\pscustom[linewidth=1,linecolor=curcolor]
{
\newpath
\moveto(228.794473,25.76199746)
\curveto(277.955233,-10.91621254)(327.115983,-4.78593254)(376.276743,25.76199746)
\curveto(329.051513,52.97756746)(281.323823,70.35691746)(228.794473,25.76199746)
\closepath
}
}
{
\newrgbcolor{curcolor}{0 0 0}
\pscustom[linewidth=1,linecolor=curcolor]
{
\newpath
\moveto(400.520403,158.09197746)
\lineto(400.520403,52.02595746)
}
}
{
\newrgbcolor{curcolor}{0 0 0}
\pscustom[linewidth=1,linecolor=curcolor]
{
\newpath
\moveto(51.007622,185.36609746)
\curveto(112.776333,228.83284746)(141.907863,200.93198746)(178.286843,185.36609746)
\curveto(145.048353,164.56948746)(113.042273,140.74129746)(51.007622,185.36609746)
\closepath
}
}
{
\newrgbcolor{curcolor}{0 0 0}
\pscustom[linewidth=1,linecolor=curcolor]
{
\newpath
\moveto(228.794473,185.36609746)
\curveto(269.817823,210.94079746)(316.468463,218.82973746)(376.276743,185.36609746)
\curveto(332.003653,154.81816746)(283.823743,148.68788746)(228.794473,185.36609746)
\closepath
}
}
{
\newrgbcolor{curcolor}{0 0 0}
\pscustom[linewidth=1,linecolor=curcolor]
{
\newpath
\moveto(228.794473,185.36609746)
\lineto(376.276743,185.36609746)
}
}
{
\newrgbcolor{curcolor}{0 0 0}
\pscustom[linestyle=none,fillstyle=solid,fillcolor=curcolor]
{
\newpath
\moveto(93.70706364,216.15466622)
\lineto(96.60745426,216.15466622)
\lineto(96.60745426,226.1654084)
\lineto(93.45218082,225.5325959)
\lineto(93.45218082,227.1497834)
\lineto(96.58987614,227.7825959)
\lineto(98.36526676,227.7825959)
\lineto(98.36526676,216.15466622)
\lineto(101.26565739,216.15466622)
\lineto(101.26565739,214.66052559)
\lineto(93.70706364,214.66052559)
\lineto(93.70706364,216.15466622)
}
}
{
\newrgbcolor{curcolor}{0 0 0}
\pscustom[linestyle=none,fillstyle=solid,fillcolor=curcolor]
{
\newpath
\moveto(105.16800114,216.15466622)
\lineto(108.06839176,216.15466622)
\lineto(108.06839176,226.1654084)
\lineto(104.91311832,225.5325959)
\lineto(104.91311832,227.1497834)
\lineto(108.05081364,227.7825959)
\lineto(109.82620426,227.7825959)
\lineto(109.82620426,216.15466622)
\lineto(112.72659489,216.15466622)
\lineto(112.72659489,214.66052559)
\lineto(105.16800114,214.66052559)
\lineto(105.16800114,216.15466622)
}
}
{
\newrgbcolor{curcolor}{0 0 0}
\pscustom[linestyle=none,fillstyle=solid,fillcolor=curcolor]
{
\newpath
\moveto(116.50589176,223.96814278)
\lineto(118.36038395,223.96814278)
\lineto(118.36038395,221.7357209)
\lineto(116.50589176,221.7357209)
\lineto(116.50589176,223.96814278)
\moveto(116.50589176,216.89294747)
\lineto(118.36038395,216.89294747)
\lineto(118.36038395,215.38122872)
\lineto(116.9189777,212.56872872)
\lineto(115.78518864,212.56872872)
\lineto(116.50589176,215.38122872)
\lineto(116.50589176,216.89294747)
}
}
{
\newrgbcolor{curcolor}{0 0 0}
\pscustom[linestyle=none,fillstyle=solid,fillcolor=curcolor]
{
\newpath
\moveto(126.41995426,221.92908028)
\curveto(125.62307412,221.92907301)(124.99026225,221.65661234)(124.52151676,221.11169747)
\curveto(124.05862256,220.56676968)(123.82717748,219.81970012)(123.82718082,218.87048653)
\curveto(123.82717748,217.92712389)(124.05862256,217.18005432)(124.52151676,216.62927559)
\curveto(124.99026225,216.08435229)(125.62307412,215.81189163)(126.41995426,215.81189278)
\curveto(127.21682252,215.81189163)(127.84670471,216.08435229)(128.3096027,216.62927559)
\curveto(128.7783444,217.18005432)(129.01271917,217.92712389)(129.0127277,218.87048653)
\curveto(129.01271917,219.81970012)(128.7783444,220.56676968)(128.3096027,221.11169747)
\curveto(127.84670471,221.65661234)(127.21682252,221.92907301)(126.41995426,221.92908028)
\moveto(129.94436832,227.49255684)
\lineto(129.94436832,225.87536934)
\curveto(129.4990468,226.08629542)(129.04787538,226.24742807)(128.5908527,226.35876778)
\curveto(128.13967316,226.47008409)(127.69143142,226.5257481)(127.24612614,226.52575997)
\curveto(126.07424554,226.5257481)(125.17776206,226.13024068)(124.55667301,225.33923653)
\curveto(123.94143517,224.54821102)(123.58987303,223.35289971)(123.50198551,221.75329903)
\curveto(123.84768527,222.26305705)(124.28127858,222.6527051)(124.80276676,222.92224434)
\curveto(125.32424629,223.19762643)(125.89846447,223.3353216)(126.52542301,223.33533028)
\curveto(127.84377502,223.3353216)(128.88381304,222.93395482)(129.6455402,222.13122872)
\curveto(130.41310839,221.33434704)(130.79689707,220.24743407)(130.79690739,218.87048653)
\curveto(130.79689707,217.52282742)(130.39845997,216.44177381)(129.60159489,215.62732247)
\curveto(128.80471156,214.81286919)(127.74416575,214.40564303)(126.41995426,214.40564278)
\curveto(124.90237171,214.40564303)(123.74221662,214.98572058)(122.93948551,216.14587715)
\curveto(122.13674948,217.31189013)(121.73538269,218.99938844)(121.73538395,221.20837715)
\curveto(121.73538269,223.28258728)(122.2275697,224.93492938)(123.21194645,226.1654084)
\curveto(124.19631773,227.40172379)(125.51760547,228.01988723)(127.17581364,228.01990059)
\curveto(127.62111899,228.01988723)(128.06936073,227.97594196)(128.5205402,227.88806465)
\curveto(128.97756295,227.80016089)(129.45217185,227.66832508)(129.94436832,227.49255684)
}
}
{
\newrgbcolor{curcolor}{0 0 0}
\pscustom[linestyle=none,fillstyle=solid,fillcolor=curcolor]
{
\newpath
\moveto(92.18914219,171.12492684)
\lineto(95.08953281,171.12492684)
\lineto(95.08953281,181.13566902)
\lineto(91.93425938,180.50285652)
\lineto(91.93425938,182.12004402)
\lineto(95.07195469,182.75285652)
\lineto(96.84734531,182.75285652)
\lineto(96.84734531,171.12492684)
\lineto(99.74773594,171.12492684)
\lineto(99.74773594,169.63078621)
\lineto(92.18914219,169.63078621)
\lineto(92.18914219,171.12492684)
}
}
{
\newrgbcolor{curcolor}{0 0 0}
\pscustom[linestyle=none,fillstyle=solid,fillcolor=curcolor]
{
\newpath
\moveto(107.1393375,181.58391121)
\curveto(106.22527019,181.58389926)(105.53679432,181.13272783)(105.07390781,180.23039559)
\curveto(104.61687337,179.33390151)(104.38835797,177.98331692)(104.38836094,176.17863777)
\curveto(104.38835797,174.3798049)(104.61687337,173.02922031)(105.07390781,172.12687996)
\curveto(105.53679432,171.23039399)(106.22527019,170.78215225)(107.1393375,170.7821534)
\curveto(108.05925274,170.78215225)(108.74772861,171.23039399)(109.20476719,172.12687996)
\curveto(109.66764956,173.02922031)(109.89909465,174.3798049)(109.89910313,176.17863777)
\curveto(109.89909465,177.98331692)(109.66764956,179.33390151)(109.20476719,180.23039559)
\curveto(108.74772861,181.13272783)(108.05925274,181.58389926)(107.1393375,181.58391121)
\moveto(107.1393375,182.99016121)
\curveto(108.61003343,182.99014785)(109.73210263,182.40714062)(110.50554844,181.24113777)
\curveto(111.28483545,180.08097107)(111.6744835,178.39347276)(111.67449375,176.17863777)
\curveto(111.6744835,173.96964906)(111.28483545,172.28215075)(110.50554844,171.11613777)
\curveto(109.73210263,169.9559812)(108.61003343,169.37590365)(107.1393375,169.3759034)
\curveto(105.66863013,169.37590365)(104.54363125,169.9559812)(103.7643375,171.11613777)
\curveto(102.99089843,172.28215075)(102.60418007,173.96964906)(102.60418125,176.17863777)
\curveto(102.60418007,178.39347276)(102.99089843,180.08097107)(103.7643375,181.24113777)
\curveto(104.54363125,182.40714062)(105.66863013,182.99014785)(107.1393375,182.99016121)
}
}
{
\newrgbcolor{curcolor}{0 0 0}
\pscustom[linestyle=none,fillstyle=solid,fillcolor=curcolor]
{
\newpath
\moveto(114.98797031,178.9384034)
\lineto(116.8424625,178.9384034)
\lineto(116.8424625,176.70598152)
\lineto(114.98797031,176.70598152)
\lineto(114.98797031,178.9384034)
\moveto(114.98797031,171.86320809)
\lineto(116.8424625,171.86320809)
\lineto(116.8424625,170.35148934)
\lineto(115.40105625,167.53898934)
\lineto(114.26726719,167.53898934)
\lineto(114.98797031,170.35148934)
\lineto(114.98797031,171.86320809)
}
}
{
\newrgbcolor{curcolor}{0 0 0}
\pscustom[linestyle=none,fillstyle=solid,fillcolor=curcolor]
{
\newpath
\moveto(120.90300938,182.75285652)
\lineto(127.87273594,182.75285652)
\lineto(127.87273594,181.2587159)
\lineto(122.52898594,181.2587159)
\lineto(122.52898594,178.04191902)
\curveto(122.78679461,178.12980115)(123.04460686,178.19425421)(123.30242344,178.2352784)
\curveto(123.56023134,178.28214475)(123.81804358,178.30558222)(124.07586094,178.3055909)
\curveto(125.54069811,178.30558222)(126.7008532,177.90421544)(127.55632969,177.10148934)
\curveto(128.41178899,176.29874829)(128.83952294,175.21183532)(128.83953281,173.84074715)
\curveto(128.83952294,172.42863498)(128.40007025,171.33000326)(127.52117344,170.54484871)
\curveto(126.64225951,169.7655517)(125.40300293,169.37590365)(123.8034,169.3759034)
\curveto(123.25261446,169.37590365)(122.69011502,169.42277861)(122.1159,169.5165284)
\curveto(121.54753804,169.61027842)(120.95867144,169.75090328)(120.34929844,169.9384034)
\lineto(120.34929844,171.72258309)
\curveto(120.87664027,171.43547191)(121.4215616,171.22160493)(121.98406406,171.08098152)
\curveto(122.54656048,170.94035521)(123.14128645,170.87004278)(123.76824375,170.87004402)
\curveto(124.78190981,170.87004278)(125.58464338,171.13664408)(126.17644688,171.66984871)
\curveto(126.76823594,172.20304926)(127.06413409,172.92668135)(127.06414219,173.84074715)
\curveto(127.06413409,174.75480452)(126.76823594,175.47843661)(126.17644688,176.01164559)
\curveto(125.58464338,176.5448418)(124.78190981,176.81144309)(123.76824375,176.81145027)
\curveto(123.29363004,176.81144309)(122.81902114,176.75870877)(122.34441563,176.65324715)
\curveto(121.87566271,176.54777148)(121.39519444,176.38370915)(120.90300938,176.16105965)
\lineto(120.90300938,182.75285652)
}
}
{
\newrgbcolor{curcolor}{0 0 0}
\pscustom[linestyle=none,fillstyle=solid,fillcolor=curcolor]
{
\newpath
\moveto(288.41301914,226.23001412)
\lineto(296.85051914,226.23001412)
\lineto(296.85051914,225.47415474)
\lineto(292.08684727,213.1079438)
\lineto(290.23235508,213.1079438)
\lineto(294.71477696,224.73587349)
\lineto(288.41301914,224.73587349)
\lineto(288.41301914,226.23001412)
}
}
{
\newrgbcolor{curcolor}{0 0 0}
\pscustom[linestyle=none,fillstyle=solid,fillcolor=curcolor]
{
\newpath
\moveto(300.50676914,222.41556099)
\lineto(302.36126133,222.41556099)
\lineto(302.36126133,220.18313912)
\lineto(300.50676914,220.18313912)
\lineto(300.50676914,222.41556099)
\moveto(300.50676914,215.34036568)
\lineto(302.36126133,215.34036568)
\lineto(302.36126133,213.82864693)
\lineto(300.91985508,211.01614693)
\lineto(299.78606602,211.01614693)
\lineto(300.50676914,213.82864693)
\lineto(300.50676914,215.34036568)
}
}
{
\newrgbcolor{curcolor}{0 0 0}
\pscustom[linestyle=none,fillstyle=solid,fillcolor=curcolor]
{
\newpath
\moveto(307.93352696,214.60208443)
\lineto(314.12981602,214.60208443)
\lineto(314.12981602,213.1079438)
\lineto(305.79778477,213.1079438)
\lineto(305.79778477,214.60208443)
\curveto(306.4716109,215.29934786)(307.38860217,216.23391724)(308.54876133,217.40579537)
\curveto(309.71477172,218.58352427)(310.44719286,219.34231257)(310.74602696,219.68216255)
\curveto(311.3143795,220.32082722)(311.70988691,220.85988918)(311.93255039,221.29935005)
\curveto(312.16105834,221.74465392)(312.27531603,222.18117692)(312.27532383,222.60892037)
\curveto(312.27531603,223.30617579)(312.02922253,223.8745346)(311.53704258,224.31399849)
\curveto(311.05070788,224.75343997)(310.41496633,224.97316631)(309.62981602,224.97317818)
\curveto(309.0731708,224.97316631)(308.4843042,224.87648672)(307.86321446,224.68313912)
\curveto(307.24797731,224.48976836)(306.58879828,224.1967999)(305.88567539,223.80423287)
\lineto(305.88567539,225.59720162)
\curveto(306.60051702,225.88429821)(307.2684851,226.10109487)(307.88958164,226.24759224)
\curveto(308.51067136,226.39406333)(309.07903017,226.46730544)(309.59465977,226.4673188)
\curveto(310.95402829,226.46730544)(312.03801158,226.12746203)(312.84661289,225.44778755)
\curveto(313.65519747,224.76808839)(314.05949394,223.85988618)(314.05950352,222.72317818)
\curveto(314.05949394,222.1841066)(313.95695498,221.6714118)(313.75188633,221.18509224)
\curveto(313.55265851,220.70461589)(313.18644794,220.13625709)(312.65325352,219.48001412)
\curveto(312.50676112,219.31008604)(312.04094127,218.81789903)(311.25579258,218.00345162)
\curveto(310.47063034,217.19485378)(309.36320957,216.06106585)(307.93352696,214.60208443)
}
}
{
\newrgbcolor{curcolor}{0 0 0}
\pscustom[linestyle=none,fillstyle=solid,fillcolor=curcolor]
{
\newpath
\moveto(293.13189121,197.02116707)
\curveto(292.33501107,197.0211598)(291.7021992,196.74869914)(291.23345371,196.20378426)
\curveto(290.77055951,195.65885648)(290.53911443,194.91178691)(290.53911778,193.96257332)
\curveto(290.53911443,193.01921068)(290.77055951,192.27214111)(291.23345371,191.72136238)
\curveto(291.7021992,191.17643908)(292.33501107,190.90397842)(293.13189121,190.90397957)
\curveto(293.92875947,190.90397842)(294.55864166,191.17643908)(295.02153965,191.72136238)
\curveto(295.49028135,192.27214111)(295.72465612,193.01921068)(295.72466465,193.96257332)
\curveto(295.72465612,194.91178691)(295.49028135,195.65885648)(295.02153965,196.20378426)
\curveto(294.55864166,196.74869914)(293.92875947,197.0211598)(293.13189121,197.02116707)
\moveto(296.65630528,202.58464363)
\lineto(296.65630528,200.96745613)
\curveto(296.21098375,201.17838221)(295.75981233,201.33951486)(295.30278965,201.45085457)
\curveto(294.85161011,201.56217089)(294.40336838,201.61783489)(293.95806309,201.61784676)
\curveto(292.78618249,201.61783489)(291.88969901,201.22232748)(291.26860996,200.43132332)
\curveto(290.65337213,199.64029781)(290.30180998,198.4449865)(290.21392246,196.84538582)
\curveto(290.55962222,197.35514384)(290.99321554,197.74479189)(291.51470371,198.01433113)
\curveto(292.03618324,198.28971322)(292.61040142,198.4274084)(293.23735996,198.42741707)
\curveto(294.55571197,198.4274084)(295.59575,198.02604161)(296.35747715,197.22331551)
\curveto(297.12504534,196.42643383)(297.50883402,195.33952086)(297.50884434,193.96257332)
\curveto(297.50883402,192.61491421)(297.11039692,191.5338606)(296.31353184,190.71940926)
\curveto(295.51664851,189.90495598)(294.4561027,189.49772983)(293.13189121,189.49772957)
\curveto(291.61430866,189.49772983)(290.45415357,190.07780737)(289.65142246,191.23796395)
\curveto(288.84868643,192.40397692)(288.44731964,194.09147523)(288.4473209,196.30046395)
\curveto(288.44731964,198.37467407)(288.93950665,200.02701617)(289.9238834,201.2574952)
\curveto(290.90825468,202.49381058)(292.22954242,203.11197402)(293.88775059,203.11198738)
\curveto(294.33305595,203.11197402)(294.78129768,203.06802875)(295.23247715,202.98015145)
\curveto(295.6894999,202.89224768)(296.1641088,202.76041188)(296.65630528,202.58464363)
}
}
{
\newrgbcolor{curcolor}{0 0 0}
\pscustom[linestyle=none,fillstyle=solid,fillcolor=curcolor]
{
\newpath
\moveto(300.76079746,199.06022957)
\lineto(302.61528965,199.06022957)
\lineto(302.61528965,196.8278077)
\lineto(300.76079746,196.8278077)
\lineto(300.76079746,199.06022957)
\moveto(300.76079746,191.98503426)
\lineto(302.61528965,191.98503426)
\lineto(302.61528965,190.47331551)
\lineto(301.1738834,187.66081551)
\lineto(300.04009434,187.66081551)
\lineto(300.76079746,190.47331551)
\lineto(300.76079746,191.98503426)
}
}
{
\newrgbcolor{curcolor}{0 0 0}
\pscustom[linestyle=none,fillstyle=solid,fillcolor=curcolor]
{
\newpath
\moveto(306.96587559,191.24675301)
\lineto(309.86626621,191.24675301)
\lineto(309.86626621,201.2574952)
\lineto(306.71099278,200.6246827)
\lineto(306.71099278,202.2418702)
\lineto(309.84868809,202.8746827)
\lineto(311.62407871,202.8746827)
\lineto(311.62407871,191.24675301)
\lineto(314.52446934,191.24675301)
\lineto(314.52446934,189.75261238)
\lineto(306.96587559,189.75261238)
\lineto(306.96587559,191.24675301)
}
}
{
\newrgbcolor{curcolor}{0 0 0}
\pscustom[linestyle=none,fillstyle=solid,fillcolor=curcolor]
{
\newpath
\moveto(294.53374668,172.66536263)
\curveto(295.3833479,172.48371511)(296.04545662,172.1057858)(296.52007481,171.53157356)
\curveto(297.00053379,170.95734945)(297.24076792,170.24836578)(297.24077793,169.40462044)
\curveto(297.24076792,168.10969604)(296.79545587,167.10774392)(295.90484043,166.39876106)
\curveto(295.01420765,165.68977659)(293.74858391,165.33528476)(292.10796543,165.3352845)
\curveto(291.55717985,165.33528476)(290.98882105,165.39094876)(290.40288731,165.50227669)
\curveto(289.82280659,165.60774542)(289.22222125,165.76887807)(288.60112949,165.98567513)
\lineto(288.60112949,167.69954231)
\curveto(289.09331513,167.41243112)(289.63237709,167.19563446)(290.21831699,167.04915169)
\curveto(290.80425092,166.902666)(291.416555,166.82942389)(292.05523106,166.82942513)
\curveto(293.16850637,166.82942389)(294.01518521,167.04915023)(294.59527012,167.48860481)
\curveto(295.18119967,167.9280556)(295.47416813,168.56672684)(295.47417637,169.40462044)
\curveto(295.47416813,170.17805335)(295.20170746,170.78156837)(294.65679356,171.21516731)
\curveto(294.11772417,171.65461438)(293.36479523,171.87434072)(292.39800449,171.874347)
\lineto(290.86870762,171.874347)
\lineto(290.86870762,173.33333138)
\lineto(292.46831699,173.33333138)
\curveto(293.34135776,173.33332363)(294.00932584,173.50617502)(294.47222324,173.85188606)
\curveto(294.93510616,174.20343995)(295.16655125,174.7073457)(295.16655918,175.36360481)
\curveto(295.16655125,176.03742249)(294.92631711,176.55304698)(294.44585606,176.91047981)
\curveto(293.97123994,177.27374938)(293.28862344,177.45538982)(292.39800449,177.45540169)
\curveto(291.91167169,177.45538982)(291.39018783,177.4026555)(290.83355137,177.29719856)
\curveto(290.2769077,177.19171821)(289.66460362,177.02765588)(288.99663731,176.80501106)
\lineto(288.99663731,178.38704231)
\curveto(289.67046299,178.57452933)(290.30034517,178.71515419)(290.88628574,178.80891731)
\curveto(291.47807837,178.902654)(292.03471844,178.94952896)(292.55620762,178.94954231)
\curveto(293.9038572,178.94952896)(294.97026238,178.64191208)(295.75542637,178.02669075)
\curveto(296.54057331,177.41730392)(296.93315104,176.59113288)(296.93316074,175.54817513)
\curveto(296.93315104,174.8216034)(296.72514344,174.20636964)(296.30913731,173.702472)
\curveto(295.89311302,173.20441751)(295.30131674,172.85871473)(294.53374668,172.66536263)
}
}
{
\newrgbcolor{curcolor}{0 0 0}
\pscustom[linestyle=none,fillstyle=solid,fillcolor=curcolor]
{
\newpath
\moveto(300.80034824,174.8977845)
\lineto(302.65484043,174.8977845)
\lineto(302.65484043,172.66536263)
\lineto(300.80034824,172.66536263)
\lineto(300.80034824,174.8977845)
\moveto(300.80034824,167.82258919)
\lineto(302.65484043,167.82258919)
\lineto(302.65484043,166.31087044)
\lineto(301.21343418,163.49837044)
\lineto(300.07964512,163.49837044)
\lineto(300.80034824,166.31087044)
\lineto(300.80034824,167.82258919)
}
}
{
\newrgbcolor{curcolor}{0 0 0}
\pscustom[linestyle=none,fillstyle=solid,fillcolor=curcolor]
{
\newpath
\moveto(311.57573887,177.16536263)
\lineto(307.09331699,170.16047981)
\lineto(311.57573887,170.16047981)
\lineto(311.57573887,177.16536263)
\moveto(311.10991856,178.71223763)
\lineto(313.34234043,178.71223763)
\lineto(313.34234043,170.16047981)
\lineto(315.21441074,170.16047981)
\lineto(315.21441074,168.68391731)
\lineto(313.34234043,168.68391731)
\lineto(313.34234043,165.59016731)
\lineto(311.57573887,165.59016731)
\lineto(311.57573887,168.68391731)
\lineto(305.65191074,168.68391731)
\lineto(305.65191074,170.3977845)
\lineto(311.10991856,178.71223763)
}
}
{
\newrgbcolor{curcolor}{0 0 0}
\pscustom[linestyle=none,fillstyle=solid,fillcolor=curcolor]
{
\newpath
\moveto(405.2476749,101.42439583)
\lineto(408.14806553,101.42439583)
\lineto(408.14806553,111.43513802)
\lineto(404.99279209,110.80232552)
\lineto(404.99279209,112.41951302)
\lineto(408.1304874,113.05232552)
\lineto(409.90587803,113.05232552)
\lineto(409.90587803,101.42439583)
\lineto(412.80626865,101.42439583)
\lineto(412.80626865,99.93025521)
\lineto(405.2476749,99.93025521)
\lineto(405.2476749,101.42439583)
}
}
{
\newrgbcolor{curcolor}{0 0 0}
\pscustom[linestyle=none,fillstyle=solid,fillcolor=curcolor]
{
\newpath
\moveto(416.58556553,109.23787239)
\lineto(418.44005772,109.23787239)
\lineto(418.44005772,107.00545052)
\lineto(416.58556553,107.00545052)
\lineto(416.58556553,109.23787239)
\moveto(416.58556553,102.16267708)
\lineto(418.44005772,102.16267708)
\lineto(418.44005772,100.65095833)
\lineto(416.99865147,97.83845833)
\lineto(415.8648624,97.83845833)
\lineto(416.58556553,100.65095833)
\lineto(416.58556553,102.16267708)
}
}
{
\newrgbcolor{curcolor}{0 0 0}
\pscustom[linestyle=none,fillstyle=solid,fillcolor=curcolor]
{
\newpath
\moveto(426.27990147,106.16170052)
\curveto(425.43614659,106.16169429)(424.77110819,105.93610857)(424.28478428,105.48494271)
\curveto(423.80431228,105.03376573)(423.56407815,104.4126726)(423.56408115,103.62166146)
\curveto(423.56407815,102.83064293)(423.80431228,102.2095498)(424.28478428,101.75838021)
\curveto(424.77110819,101.30720695)(425.43614659,101.08162124)(426.27990147,101.08162239)
\curveto(427.1236449,101.08162124)(427.7886833,101.30720695)(428.27501865,101.75838021)
\curveto(428.76133858,102.21540917)(429.0045024,102.8365023)(429.00451084,103.62166146)
\curveto(429.0045024,104.4126726)(428.76133858,105.03376573)(428.27501865,105.48494271)
\curveto(427.79454267,105.93610857)(427.12950427,106.16169429)(426.27990147,106.16170052)
\moveto(424.50451084,106.91755989)
\curveto(423.74278891,107.10505272)(423.14806294,107.45954455)(422.72033115,107.98103646)
\curveto(422.29845441,108.50251226)(422.08751713,109.13825381)(422.08751865,109.88826302)
\curveto(422.08751713,110.93708014)(422.45958707,111.76618087)(423.20372959,112.37556771)
\curveto(423.9537262,112.98492965)(424.9791158,113.28961685)(426.27990147,113.28963021)
\curveto(427.58653506,113.28961685)(428.61192466,112.98492965)(429.35607334,112.37556771)
\curveto(430.10020442,111.76618087)(430.47227437,110.93708014)(430.47228428,109.88826302)
\curveto(430.47227437,109.13825381)(430.25840739,108.50251226)(429.83068272,107.98103646)
\curveto(429.40879887,107.45954455)(428.81993227,107.10505272)(428.06408115,106.91755989)
\curveto(428.91954154,106.71833435)(429.58457994,106.32868631)(430.05919834,105.74861458)
\curveto(430.53965711,105.16853122)(430.77989125,104.45954755)(430.77990147,103.62166146)
\curveto(430.77989125,102.35017466)(430.3902432,101.3745897)(429.61095615,100.69490364)
\curveto(428.83751038,100.01521606)(427.72715992,99.67537265)(426.27990147,99.67537239)
\curveto(424.83263157,99.67537265)(423.71935143,100.01521606)(422.94005772,100.69490364)
\curveto(422.16661861,101.3745897)(421.77990025,102.35017466)(421.77990147,103.62166146)
\curveto(421.77990025,104.45954755)(422.02013438,105.16853122)(422.50060459,105.74861458)
\curveto(422.98107092,106.32868631)(423.649039,106.71833435)(424.50451084,106.91755989)
\moveto(423.85412022,109.72127083)
\curveto(423.85411692,109.04157422)(424.06505421,108.51130131)(424.48693272,108.13045052)
\curveto(424.91466274,107.74958332)(425.51231839,107.55915383)(426.27990147,107.55916146)
\curveto(427.04161373,107.55915383)(427.6363397,107.74958332)(428.06408115,108.13045052)
\curveto(428.49766696,108.51130131)(428.71446362,109.04157422)(428.71447178,109.72127083)
\curveto(428.71446362,110.40094786)(428.49766696,110.93122077)(428.06408115,111.31209114)
\curveto(427.6363397,111.69293875)(427.04161373,111.88336825)(426.27990147,111.88338021)
\curveto(425.51231839,111.88336825)(424.91466274,111.69293875)(424.48693272,111.31209114)
\curveto(424.06505421,110.93122077)(423.85411692,110.40094786)(423.85412022,109.72127083)
}
}
{
\newrgbcolor{curcolor}{0 0 0}
\pscustom[linestyle=none,fillstyle=solid,fillcolor=curcolor]
{
\newpath
\moveto(234.59282871,113.05232552)
\lineto(243.03032871,113.05232552)
\lineto(243.03032871,112.29646614)
\lineto(238.26665684,99.93025521)
\lineto(236.41216465,99.93025521)
\lineto(240.89458653,111.55818489)
\lineto(234.59282871,111.55818489)
\lineto(234.59282871,113.05232552)
}
}
{
\newrgbcolor{curcolor}{0 0 0}
\pscustom[linestyle=none,fillstyle=solid,fillcolor=curcolor]
{
\newpath
\moveto(246.68657871,109.23787239)
\lineto(248.5410709,109.23787239)
\lineto(248.5410709,107.00545052)
\lineto(246.68657871,107.00545052)
\lineto(246.68657871,109.23787239)
\moveto(246.68657871,102.16267708)
\lineto(248.5410709,102.16267708)
\lineto(248.5410709,100.65095833)
\lineto(247.09966465,97.83845833)
\lineto(245.96587559,97.83845833)
\lineto(246.68657871,100.65095833)
\lineto(246.68657871,102.16267708)
}
}
{
\newrgbcolor{curcolor}{0 0 0}
\pscustom[linestyle=none,fillstyle=solid,fillcolor=curcolor]
{
\newpath
\moveto(252.63677403,100.20271614)
\lineto(252.63677403,101.81990364)
\curveto(253.0820841,101.60896446)(253.53325553,101.44783181)(253.99028965,101.33650521)
\curveto(254.44731711,101.22517579)(254.89555885,101.16951178)(255.33501621,101.16951302)
\curveto(256.50688537,101.16951178)(257.40043916,101.56208951)(258.01568028,102.34724739)
\curveto(258.63676605,103.13825981)(258.99125788,104.3365008)(259.07915684,105.94197396)
\curveto(258.73930501,105.4380622)(258.30864138,105.05134383)(257.78716465,104.78181771)
\curveto(257.26567367,104.51228187)(256.68852581,104.37751638)(256.05571934,104.37752083)
\curveto(254.74321525,104.37751638)(253.70317723,104.7730238)(252.93560215,105.56404427)
\curveto(252.17388189,106.36091284)(251.79302289,107.44782581)(251.79302403,108.82478646)
\curveto(251.79302289,110.17243246)(252.19145999,111.25348607)(252.98833653,112.06795052)
\curveto(253.7852084,112.88239069)(254.84575421,113.28961685)(256.16997715,113.28963021)
\curveto(257.68754825,113.28961685)(258.84477365,112.70660962)(259.64165684,111.54060677)
\curveto(260.44438143,110.38044007)(260.84574821,108.69294175)(260.8457584,106.47810677)
\curveto(260.84574821,104.40974291)(260.35356121,102.75740082)(259.3691959,101.52107552)
\curveto(258.39067254,100.29060641)(257.07231449,99.67537265)(255.41411778,99.67537239)
\curveto(254.96880097,99.67537265)(254.51762954,99.71931792)(254.06060215,99.80720833)
\curveto(253.60356796,99.89509899)(253.12895906,100.0269348)(252.63677403,100.20271614)
\moveto(256.16997715,105.76619271)
\curveto(256.96684584,105.76618687)(257.59672803,106.03864753)(258.05962559,106.58357552)
\curveto(258.52836772,107.12849019)(258.76274248,107.87555976)(258.76275059,108.82478646)
\curveto(258.76274248,109.76813599)(258.52836772,110.51227587)(258.05962559,111.05720833)
\curveto(257.59672803,111.6079779)(256.96684584,111.88336825)(256.16997715,111.88338021)
\curveto(255.37309744,111.88336825)(254.74028557,111.6079779)(254.27153965,111.05720833)
\curveto(253.80864588,110.51227587)(253.57720079,109.76813599)(253.57720371,108.82478646)
\curveto(253.57720079,107.87555976)(253.80864588,107.12849019)(254.27153965,106.58357552)
\curveto(254.74028557,106.03864753)(255.37309744,105.76618687)(256.16997715,105.76619271)
}
}
{
\newrgbcolor{curcolor}{0 0 0}
\pscustom[linestyle=none,fillstyle=solid,fillcolor=curcolor]
{
\newpath
\moveto(187.46640965,106.18200997)
\curveto(186.62265477,106.18200373)(185.95761637,105.95641802)(185.47129246,105.50525215)
\curveto(184.99082046,105.05407517)(184.75058633,104.43298205)(184.75058933,103.6419709)
\curveto(184.75058633,102.85095238)(184.99082046,102.22985925)(185.47129246,101.77868965)
\curveto(185.95761637,101.3275164)(186.62265477,101.10193069)(187.46640965,101.10193184)
\curveto(188.31015308,101.10193069)(188.97519148,101.3275164)(189.46152683,101.77868965)
\curveto(189.94784675,102.23571862)(190.19101057,102.85681175)(190.19101902,103.6419709)
\curveto(190.19101057,104.43298205)(189.94784675,105.05407517)(189.46152683,105.50525215)
\curveto(188.98105085,105.95641802)(188.31601245,106.18200373)(187.46640965,106.18200997)
\moveto(185.69101902,106.93786934)
\curveto(184.92929709,107.12536217)(184.33457112,107.479854)(183.90683933,108.0013459)
\curveto(183.48496259,108.52282171)(183.2740253,109.15856326)(183.27402683,109.90857247)
\curveto(183.2740253,110.95738958)(183.64609524,111.78649032)(184.39023777,112.39587715)
\curveto(185.14023438,113.0052391)(186.16562397,113.30992629)(187.46640965,113.30993965)
\curveto(188.77304324,113.30992629)(189.79843284,113.0052391)(190.54258152,112.39587715)
\curveto(191.2867126,111.78649032)(191.65878254,110.95738958)(191.65879246,109.90857247)
\curveto(191.65878254,109.15856326)(191.44491557,108.52282171)(191.0171909,108.0013459)
\curveto(190.59530704,107.479854)(190.00644045,107.12536217)(189.25058933,106.93786934)
\curveto(190.10604972,106.7386438)(190.77108812,106.34899575)(191.24570652,105.76892403)
\curveto(191.72616529,105.18884067)(191.96639942,104.479857)(191.96640965,103.6419709)
\curveto(191.96639942,102.37048411)(191.57675138,101.39489915)(190.79746433,100.71521309)
\curveto(190.02401855,100.03552551)(188.9136681,99.6956821)(187.46640965,99.69568184)
\curveto(186.01913975,99.6956821)(184.90585961,100.03552551)(184.1265659,100.71521309)
\curveto(183.35312679,101.39489915)(182.96640842,102.37048411)(182.96640965,103.6419709)
\curveto(182.96640842,104.479857)(183.20664256,105.18884067)(183.68711277,105.76892403)
\curveto(184.1675791,106.34899575)(184.83554718,106.7386438)(185.69101902,106.93786934)
\moveto(185.0406284,109.74158028)
\curveto(185.0406251,109.06188367)(185.25156239,108.53161076)(185.6734409,108.15075997)
\curveto(186.10117091,107.76989277)(186.69882657,107.57946327)(187.46640965,107.5794709)
\curveto(188.22812191,107.57946327)(188.82284788,107.76989277)(189.25058933,108.15075997)
\curveto(189.68417514,108.53161076)(189.9009718,109.06188367)(189.90097996,109.74158028)
\curveto(189.9009718,110.42125731)(189.68417514,110.95153021)(189.25058933,111.33240059)
\curveto(188.82284788,111.7132482)(188.22812191,111.9036777)(187.46640965,111.90368965)
\curveto(186.69882657,111.9036777)(186.10117091,111.7132482)(185.6734409,111.33240059)
\curveto(185.25156239,110.95153021)(185.0406251,110.42125731)(185.0406284,109.74158028)
}
}
{
\newrgbcolor{curcolor}{0 0 0}
\pscustom[linestyle=none,fillstyle=solid,fillcolor=curcolor]
{
\newpath
\moveto(195.31504246,109.25818184)
\lineto(197.16953465,109.25818184)
\lineto(197.16953465,107.02575997)
\lineto(195.31504246,107.02575997)
\lineto(195.31504246,109.25818184)
\moveto(195.31504246,102.18298653)
\lineto(197.16953465,102.18298653)
\lineto(197.16953465,100.67126778)
\lineto(195.7281284,97.85876778)
\lineto(194.59433933,97.85876778)
\lineto(195.31504246,100.67126778)
\lineto(195.31504246,102.18298653)
}
}
{
\newrgbcolor{curcolor}{0 0 0}
\pscustom[linestyle=none,fillstyle=solid,fillcolor=curcolor]
{
\newpath
\moveto(205.22910496,107.21911934)
\curveto(204.43222481,107.21911207)(203.79941295,106.94665141)(203.33066746,106.40173653)
\curveto(202.86777325,105.85680875)(202.63632817,105.10973918)(202.63633152,104.16052559)
\curveto(202.63632817,103.21716295)(202.86777325,102.47009338)(203.33066746,101.91931465)
\curveto(203.79941295,101.37439135)(204.43222481,101.10193069)(205.22910496,101.10193184)
\curveto(206.02597322,101.10193069)(206.6558554,101.37439135)(207.1187534,101.91931465)
\curveto(207.5874951,102.47009338)(207.82186986,103.21716295)(207.8218784,104.16052559)
\curveto(207.82186986,105.10973918)(207.5874951,105.85680875)(207.1187534,106.40173653)
\curveto(206.6558554,106.94665141)(206.02597322,107.21911207)(205.22910496,107.21911934)
\moveto(208.75351902,112.7825959)
\lineto(208.75351902,111.1654084)
\curveto(208.3081975,111.37633448)(207.85702608,111.53746713)(207.4000034,111.64880684)
\curveto(206.94882386,111.76012316)(206.50058212,111.81578716)(206.05527683,111.81579903)
\curveto(204.88339624,111.81578716)(203.98691276,111.42027975)(203.36582371,110.62927559)
\curveto(202.75058587,109.83825008)(202.39902372,108.64293877)(202.31113621,107.04333809)
\curveto(202.65683596,107.55309611)(203.09042928,107.94274416)(203.61191746,108.2122834)
\curveto(204.13339699,108.48766549)(204.70761516,108.62536067)(205.33457371,108.62536934)
\curveto(206.65292572,108.62536067)(207.69296374,108.22399388)(208.4546909,107.42126778)
\curveto(209.22225909,106.6243861)(209.60604776,105.53747313)(209.60605808,104.16052559)
\curveto(209.60604776,102.81286648)(209.20761066,101.73181287)(208.41074558,100.91736153)
\curveto(207.61386226,100.10290825)(206.55331644,99.6956821)(205.22910496,99.69568184)
\curveto(203.71152241,99.6956821)(202.55136732,100.27575964)(201.74863621,101.43591622)
\curveto(200.94590017,102.60192919)(200.54453339,104.2894275)(200.54453465,106.49841622)
\curveto(200.54453339,108.57262634)(201.0367204,110.22496844)(202.02109715,111.45544747)
\curveto(203.00546843,112.69176285)(204.32675617,113.30992629)(205.98496433,113.30993965)
\curveto(206.43026969,113.30992629)(206.87851143,113.26598103)(207.3296909,113.17810372)
\curveto(207.78671365,113.09019995)(208.26132255,112.95836415)(208.75351902,112.7825959)
}
}
{
\newrgbcolor{curcolor}{0 0 0}
\pscustom[linestyle=none,fillstyle=solid,fillcolor=curcolor]
{
\newpath
\moveto(281.30605503,62.02843025)
\lineto(287.50234409,62.02843025)
\lineto(287.50234409,60.53428963)
\lineto(279.17031284,60.53428963)
\lineto(279.17031284,62.02843025)
\curveto(279.84413898,62.72569369)(280.76113025,63.66026307)(281.92128941,64.83214119)
\curveto(283.0872998,66.00987009)(283.81972094,66.76865839)(284.11855503,67.10850838)
\curveto(284.68690757,67.74717304)(285.08241499,68.286235)(285.30507847,68.72569588)
\curveto(285.53358641,69.17099974)(285.64784411,69.60752274)(285.64785191,70.03526619)
\curveto(285.64784411,70.73252162)(285.40175061,71.30088042)(284.90957066,71.74034432)
\curveto(284.42323596,72.1797858)(283.78749441,72.39951214)(283.00234409,72.399524)
\curveto(282.44569888,72.39951214)(281.85683228,72.30283255)(281.23574253,72.10948494)
\curveto(280.62050539,71.91611418)(279.96132636,71.62314573)(279.25820347,71.23057869)
\lineto(279.25820347,73.02354744)
\curveto(279.9730451,73.31064404)(280.64101318,73.5274407)(281.26210972,73.67393807)
\curveto(281.88319944,73.82040916)(282.45155824,73.89365127)(282.96718784,73.89366463)
\curveto(284.32655637,73.89365127)(285.41053966,73.55380786)(286.21914097,72.87413338)
\curveto(287.02772554,72.19443422)(287.43202201,71.286232)(287.43203159,70.149524)
\curveto(287.43202201,69.61045243)(287.32948305,69.09775763)(287.12441441,68.61143807)
\curveto(286.92518658,68.13096172)(286.55897601,67.56260291)(286.02578159,66.90635994)
\curveto(285.87928919,66.73643186)(285.41346934,66.24424486)(284.62832066,65.42979744)
\curveto(283.84315842,64.6211996)(282.73573765,63.48741168)(281.30605503,62.02843025)
}
}
{
\newrgbcolor{curcolor}{0 0 0}
\pscustom[linestyle=none,fillstyle=solid,fillcolor=curcolor]
{
\newpath
\moveto(295.03457066,72.48741463)
\curveto(294.12050335,72.48740268)(293.43202747,72.03623125)(292.96914097,71.133899)
\curveto(292.51210652,70.23740493)(292.28359112,68.88682034)(292.28359409,67.08214119)
\curveto(292.28359112,65.28330832)(292.51210652,63.93272373)(292.96914097,63.03038338)
\curveto(293.43202747,62.1338974)(294.12050335,61.68565567)(295.03457066,61.68565682)
\curveto(295.95448589,61.68565567)(296.64296176,62.1338974)(297.10000034,63.03038338)
\curveto(297.56288272,63.93272373)(297.7943278,65.28330832)(297.79433628,67.08214119)
\curveto(297.7943278,68.88682034)(297.56288272,70.23740493)(297.10000034,71.133899)
\curveto(296.64296176,72.03623125)(295.95448589,72.48740268)(295.03457066,72.48741463)
\moveto(295.03457066,73.89366463)
\curveto(296.50526659,73.89365127)(297.62733578,73.31064404)(298.40078159,72.14464119)
\curveto(299.1800686,70.98447449)(299.56971665,69.29697618)(299.56972691,67.08214119)
\curveto(299.56971665,64.87315248)(299.1800686,63.18565417)(298.40078159,62.01964119)
\curveto(297.62733578,60.85948462)(296.50526659,60.27940707)(295.03457066,60.27940682)
\curveto(293.56386328,60.27940707)(292.43886441,60.85948462)(291.65957066,62.01964119)
\curveto(290.88613158,63.18565417)(290.49941322,64.87315248)(290.49941441,67.08214119)
\curveto(290.49941322,69.29697618)(290.88613158,70.98447449)(291.65957066,72.14464119)
\curveto(292.43886441,73.31064404)(293.56386328,73.89365127)(295.03457066,73.89366463)
}
}
{
\newrgbcolor{curcolor}{0 0 0}
\pscustom[linestyle=none,fillstyle=solid,fillcolor=curcolor]
{
\newpath
\moveto(302.88320347,69.84190682)
\lineto(304.73769566,69.84190682)
\lineto(304.73769566,67.60948494)
\lineto(302.88320347,67.60948494)
\lineto(302.88320347,69.84190682)
\moveto(302.88320347,62.7667115)
\lineto(304.73769566,62.7667115)
\lineto(304.73769566,61.25499275)
\lineto(303.29628941,58.44249275)
\lineto(302.16250034,58.44249275)
\lineto(302.88320347,61.25499275)
\lineto(302.88320347,62.7667115)
}
}
{
\newrgbcolor{curcolor}{0 0 0}
\pscustom[linestyle=none,fillstyle=solid,fillcolor=curcolor]
{
\newpath
\moveto(308.79824253,73.65635994)
\lineto(315.76796909,73.65635994)
\lineto(315.76796909,72.16221932)
\lineto(310.42421909,72.16221932)
\lineto(310.42421909,68.94542244)
\curveto(310.68202777,69.03330457)(310.93984001,69.09775763)(311.19765659,69.13878182)
\curveto(311.45546449,69.18564817)(311.71327674,69.20908564)(311.97109409,69.20909432)
\curveto(313.43593126,69.20908564)(314.59608635,68.80771886)(315.45156284,68.00499275)
\curveto(316.30702214,67.20225171)(316.73475609,66.11533874)(316.73476597,64.74425057)
\curveto(316.73475609,63.33213839)(316.2953034,62.23350668)(315.41640659,61.44835213)
\curveto(314.53749266,60.66905512)(313.29823609,60.27940707)(311.69863316,60.27940682)
\curveto(311.14784761,60.27940707)(310.58534818,60.32628202)(310.01113316,60.42003182)
\curveto(309.44277119,60.51378184)(308.8539046,60.6544067)(308.24453159,60.84190682)
\lineto(308.24453159,62.6260865)
\curveto(308.77187343,62.33897532)(309.31679476,62.12510835)(309.87929722,61.98448494)
\curveto(310.44179363,61.84385863)(311.0365196,61.7735462)(311.66347691,61.77354744)
\curveto(312.67714296,61.7735462)(313.47987653,62.0401475)(314.07168003,62.57335213)
\curveto(314.6634691,63.10655268)(314.95936724,63.83018477)(314.95937534,64.74425057)
\curveto(314.95936724,65.65830794)(314.6634691,66.38194003)(314.07168003,66.915149)
\curveto(313.47987653,67.44834521)(312.67714296,67.71494651)(311.66347691,67.71495369)
\curveto(311.1888632,67.71494651)(310.7142543,67.66221219)(310.23964878,67.55675057)
\curveto(309.77089587,67.4512749)(309.2904276,67.28721256)(308.79824253,67.06456307)
\lineto(308.79824253,73.65635994)
}
}
{
\newrgbcolor{curcolor}{0 0 0}
\pscustom[linestyle=none,fillstyle=solid,fillcolor=curcolor]
{
\newpath
\moveto(276.46438023,9.45991829)
\lineto(279.36477085,9.45991829)
\lineto(279.36477085,19.47066048)
\lineto(276.20949741,18.83784798)
\lineto(276.20949741,20.45503548)
\lineto(279.34719273,21.08784798)
\lineto(281.12258335,21.08784798)
\lineto(281.12258335,9.45991829)
\lineto(284.02297398,9.45991829)
\lineto(284.02297398,7.96577767)
\lineto(276.46438023,7.96577767)
\lineto(276.46438023,9.45991829)
}
}
{
\newrgbcolor{curcolor}{0 0 0}
\pscustom[linestyle=none,fillstyle=solid,fillcolor=curcolor]
{
\newpath
\moveto(291.41457554,19.91890267)
\curveto(290.50050823,19.91889071)(289.81203236,19.46771929)(289.34914585,18.56538704)
\curveto(288.8921114,17.66889296)(288.66359601,16.31830838)(288.66359898,14.51362923)
\curveto(288.66359601,12.71479635)(288.8921114,11.36421177)(289.34914585,10.46187142)
\curveto(289.81203236,9.56538544)(290.50050823,9.1171437)(291.41457554,9.11714485)
\curveto(292.33449077,9.1171437)(293.02296665,9.56538544)(293.48000523,10.46187142)
\curveto(293.9428876,11.36421177)(294.17433268,12.71479635)(294.17434116,14.51362923)
\curveto(294.17433268,16.31830838)(293.9428876,17.66889296)(293.48000523,18.56538704)
\curveto(293.02296665,19.46771929)(292.33449077,19.91889071)(291.41457554,19.91890267)
\moveto(291.41457554,21.32515267)
\curveto(292.88527147,21.32513931)(294.00734066,20.74213208)(294.78078648,19.57612923)
\curveto(295.56007348,18.41596253)(295.94972153,16.72846422)(295.94973179,14.51362923)
\curveto(295.94972153,12.30464051)(295.56007348,10.6171422)(294.78078648,9.45112923)
\curveto(294.00734066,8.29097265)(292.88527147,7.71089511)(291.41457554,7.71089485)
\curveto(289.94386816,7.71089511)(288.81886929,8.29097265)(288.03957554,9.45112923)
\curveto(287.26613647,10.6171422)(286.8794181,12.30464051)(286.87941929,14.51362923)
\curveto(286.8794181,16.72846422)(287.26613647,18.41596253)(288.03957554,19.57612923)
\curveto(288.81886929,20.74213208)(289.94386816,21.32513931)(291.41457554,21.32515267)
}
}
{
\newrgbcolor{curcolor}{0 0 0}
\pscustom[linestyle=none,fillstyle=solid,fillcolor=curcolor]
{
\newpath
\moveto(299.26320835,17.27339485)
\lineto(301.11770054,17.27339485)
\lineto(301.11770054,15.04097298)
\lineto(299.26320835,15.04097298)
\lineto(299.26320835,17.27339485)
\moveto(299.26320835,10.19819954)
\lineto(301.11770054,10.19819954)
\lineto(301.11770054,8.68648079)
\lineto(299.67629429,5.87398079)
\lineto(298.54250523,5.87398079)
\lineto(299.26320835,8.68648079)
\lineto(299.26320835,10.19819954)
}
}
{
\newrgbcolor{curcolor}{0 0 0}
\pscustom[linestyle=none,fillstyle=solid,fillcolor=curcolor]
{
\newpath
\moveto(305.46828648,9.45991829)
\lineto(308.3686771,9.45991829)
\lineto(308.3686771,19.47066048)
\lineto(305.21340366,18.83784798)
\lineto(305.21340366,20.45503548)
\lineto(308.35109898,21.08784798)
\lineto(310.1264896,21.08784798)
\lineto(310.1264896,9.45991829)
\lineto(313.02688023,9.45991829)
\lineto(313.02688023,7.96577767)
\lineto(305.46828648,7.96577767)
\lineto(305.46828648,9.45991829)
}
}
{
\newrgbcolor{curcolor}{0 0 0}
\pscustom[linestyle=none,fillstyle=solid,fillcolor=curcolor]
{
\newpath
\moveto(322.00051304,15.04097298)
\curveto(322.85011426,14.85932546)(323.51222297,14.48139615)(323.98684116,13.90718392)
\curveto(324.46730014,13.3329598)(324.70753428,12.62397613)(324.70754429,11.78023079)
\curveto(324.70753428,10.4853064)(324.26222222,9.48335427)(323.37160679,8.77437142)
\curveto(322.480974,8.06538694)(321.21535027,7.71089511)(319.57473179,7.71089485)
\curveto(319.02394621,7.71089511)(318.45558741,7.76655912)(317.86965366,7.87788704)
\curveto(317.28957295,7.98335577)(316.68898761,8.14448842)(316.06789585,8.36128548)
\lineto(316.06789585,10.07515267)
\curveto(316.56008149,9.78804147)(317.09914345,9.57124481)(317.68508335,9.42476204)
\curveto(318.27101728,9.27827635)(318.88332135,9.20503424)(319.52199741,9.20503548)
\curveto(320.63527273,9.20503424)(321.48195157,9.42476058)(322.06203648,9.86421517)
\curveto(322.64796603,10.30366595)(322.94093448,10.94233719)(322.94094273,11.78023079)
\curveto(322.94093448,12.5536637)(322.66847382,13.15717872)(322.12355991,13.59077767)
\curveto(321.58449053,14.03022473)(320.83156159,14.24995107)(319.86477085,14.24995735)
\lineto(318.33547398,14.24995735)
\lineto(318.33547398,15.70894173)
\lineto(319.93508335,15.70894173)
\curveto(320.80812412,15.70893399)(321.4760922,15.88178537)(321.9389896,16.22749642)
\curveto(322.40187252,16.5790503)(322.6333176,17.08295605)(322.63332554,17.73921517)
\curveto(322.6333176,18.41303284)(322.39308347,18.92865733)(321.91262241,19.28609017)
\curveto(321.4380063,19.64935973)(320.75538979,19.83100018)(319.86477085,19.83101204)
\curveto(319.37843804,19.83100018)(318.85695419,19.77826585)(318.30031773,19.67280892)
\curveto(317.74367405,19.56732856)(317.13136998,19.40326623)(316.46340366,19.18062142)
\lineto(316.46340366,20.76265267)
\curveto(317.13722935,20.95013968)(317.76711153,21.09076454)(318.3530521,21.18452767)
\curveto(318.94484473,21.27826435)(319.5014848,21.32513931)(320.02297398,21.32515267)
\curveto(321.37062355,21.32513931)(322.43702874,21.01752243)(323.22219273,20.4023011)
\curveto(324.00733967,19.79291428)(324.3999174,18.96674323)(324.3999271,17.92378548)
\curveto(324.3999174,17.19721375)(324.19190979,16.58197999)(323.77590366,16.07808235)
\curveto(323.35987938,15.58002786)(322.76808309,15.23432508)(322.00051304,15.04097298)
}
}
\end{pspicture}

\caption{Représentation simplifiée de la configuration}
\end{center}
\end{figure}

\chapter{Cahier des charges}

Suite à la première rencontre avec notre encadrant, un premier cahier des charges a été évoqué. Le logiciel doit retourner la liste des différents parcours possible pour visiter une série de lieux dans l'ordre désiré par l'utilisateur. Le but est de calculer un parcours qui minimise l'insécurité et la distance à parcourir tout en maximisant l'intérêt des lieux.
\section{Besoins}
\begin{itemize}
\item Le programme devra retourner une solution de parcours viable rapidement,
\item Le logiciel permettra de retourner une nouvelle solution si on interverti des lieux ou/et si on change l'ordre des lieux à visiter.
\item Le programme doit être simple à utiliser
\item
\end{itemize}

\chapter{Spécifation}
  Dans ce chapitre on va détailler le système à développer même si c'est de façon simplifiée sans forcement allé jusqu'à tout les différents arbres hiérarchiques. 